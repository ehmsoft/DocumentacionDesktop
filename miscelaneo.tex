\chapter{Miscel\'aneos}
\label{sec:miscelaneo}

\section{Impresi\'on}

\softwareAbogadosDesktop le permite imprimir reportes de la informaci\'on almacenada en la aplicaci\'on. Para esto debe ir al men\'u archivo y seleccionar la opci\'on imprimir(Fig.\ref{fig:ArchivoImprimir}), se abrir\'a una nueva pantalla denominada asistente de impresi\'on(Fig.\ref{fig:AsistenteImpresion}). 
  
\insertImage{Varios/Menu-Archivo}{Men\'u Archivo>Imprimir}{ArchivoImprimir}
\insertImage{Varios/AsistenteImpresion}{Asistente impresi\'on}{AsistenteImpresion}

Las opciones que se incluyen para la impresi\'on son los listados de todos los  clientes, contrapartes, juzgados, procesos, actuaciones y eventos pr\'oximos. Para esto solo debe seleccionar la casilla en la pantalla de impresi\'on, luego especificar los elementos que quiere imprimir, presionar el bot\'on aceptar, se mostrar\'a la previsualizaci\'on y puede escoger si se imprime en un archivo PDF o en una impresosra f\'isica.
 
Tambi\'en permite imprimir la informaci\'on de un cliente, contraparte, juzgado o procesos de manera detallada. Solamente debe seleccionar la casilla en la pantalla de impresion, luego especificar el elemento que quiere imprimir, presionar el bot\'on aceptar, revisar la previsualizaci\'on y escoger si se imprime en un archivo PDF o en una impresora f\'isica.
 
Finalmente le permite imprimir el codigo QR de un proceso, este c\'odigo es una identificaci\'on \'unica de cada proceso que al escanearlo por medio de la aplicaci\'on Procesos judiciales M\'ovil le permitir\'a identificar r\'apidamente el proceso al que pertenece.

\section{Importar/Exportar}

\softwareAbogadosDesktop tiene posibilidad de exportar informaci\'on de dos formas distintas:

\subsection{Exportar CSV}

El archivo CSV o archivo de valores separados por comas, es un archivo de texto plano que contiene el listado completo de la informaci\'on que usted elija exportar. Esto es \'util si usted necesita procesar en alg\'un otro programa la informaci\'on guardada en \softwareAbogadosDesktop. Esta informaci\'on tambi\'en es accesible por cualquier programa de hoja de c\'alculo como \microsoft \excel.

Para exportar a CSV simplemente haga click en el men\'u \emph{Archivo}, luego seleccione \emph{Exportar} y finalmente presione donde dice \emph{Archivo CSV}(Fig.\ref{fig:ArchivoExportarCSV}), esto lo llevar\'a al asistente de exportaci\'on, escoja lo que desea exportar (procesos, clientes, contrapartes, juzgados, actuaciones o eventos pr\'oximos), seleccione el formato y presione \emph{Aceptar}.
   
\insertImage{Varios/MenuArchivo-Exportar-ArchivoCSV}{Men\'u Archivo>Exportar>Archivo CSV}{ArchivoExportarCSV}

\subsection{Exportar Copia de seguridad}
\label{sec:exportarBackup}
Puede guardar una copia de seguridad de toda la informaci\'on almacenada en \softwareAbogadosDesktop. Se recomienda hacer esto peri\'odicamente para evitar p\'erdida de informaci\'on. 

Para exportar la copia de seguridad simplemente haga click en el men\'u \emph{Archivo}, luego seleccione \emph{Exportar} y finalmente presione donde dice \emph{Archivo de Copia de Seguridad}(Fig.\ref{fig:ArchivoExportarBackup}), esto lo llevar\'a a una ventana de guardado, escriba un nombre para el archivo, seleccione d\'onde lo quiere guardar y presiona \emph{Aceptar} o \emph{Guardar}.

\insertImage{Varios/MenuArchivo-Exportar-CopiaSeguridad}{Men\'u Archivo>Exportar>Archivo de copia de seguridad}{ArchivoExportarBackup}

NOTA: LA COPIA DE SEGURIDAD ES ALGO SUPREMAMENTE IMPORTANTE, ASEG\'URESE DE HACERLO AL MENOS UNA VEZ A LA SEMANA PARA EVITAR P\'ERDIDA DE DATOS. TAMBI\'EN SE RECOMIENDA GUARDAR EL ARCHIVO EN UN LUGAR SEGURO, COMO EN ALGUNA MEMORIA USB, O PUEDE ENVIARLO A SU CORREO ELECTR\'ONICO.

\subsection{Importar Copia de seguridad}

Puede importar una archivo de copia de seguridad que haya realizado con anterioridad. Aseg\'urese de solamente importar archivos que hayan sido exportados con anterioridad mediante \ref{sec:exportarBackup}.

NOTA: AL REALIZAR ESTA OPERACI\'ON E IMPORTAR UNA COPIA DE SEGURIDAD ANTERIOR, SE BORRAR\'AN TODOS LOS DATOS QUE HAYAN ACTUALMENTE EN EL PROGRAMA Y SE REEMPLAZAR\'AN POR LOS DATOS DEL ARCHIVO QUE EST\'A IMPORTANDO.

Para importar dir\'ijase al men\'u \emph{Archivo}, y presione \emph{Importar}(Fig.\ref{fig:ArchivoImportar}), luego aparecer\'a una ventana para que seleccione la ubicaci\'on del archivo y finalmente presione el bot\'on \emph{Open} o \emph{Abrir}.

\insertImage{Varios/MenuArchivo-Importar}{Men\'u Archivo>Importar}{ArchivoImportar}

\section{Sincronizaci\'on}

La sincronizaci\'on es la utilidad que le permite mantener actualizada la informaci\'on entre \softwareAbogadosDesktop y \softwareAbogadosMobile. Esto significa que cualquier cambio que usted haga en alguno de los dos programas, podr\'a verse reflejado en el otro. Tenga en cuenta que \softwareAbogadosMobile es una versi\'on limitada en algunas caracter\'isticas, por lo tanto s\'olo se sincronizan los campos personalizados de procesos y plantillas.

PRECAUACI\'ON: Se recomienda antes de sincronizar hacer una copia de seguridad de la informaci\'on del programa. (Pag.\pageref{sec:exportarBackup}).

Para realizar una sincronizaci\'on seleccione en el \emph{Men\'u principal}, siga las instrucciones en pantalla y de click en el bot\'on \emph{Sincronizar} (Fig.\ref{fig:Sincronizar}). 
  
\insertImage{Varios/Sincronizar}{Pantalla Sincronizar}{Sincronizar}

NOTA: Para evitar errores, es mejor que no edite el mismo registro en ambas partes, por ejemplo, si tiene un cliente llamado John Doe, que est\'a en el programa de escritorio, y tambi\'en est\'a en el m\'ovil, cuando necesite hacer una modificaci\'on, h\'agala solamente en uno de los dos, y luego sincronice. Si hace la edici\'on en ambas partes antes de sincronizar es posible que genere informaci\'on duplicada, o incluso se pierda informaci\'on. Si tiene alguna duda por favor comun\'iquese con soporte t\'ecnico: \mbox{soporte@ehmsoft.com}.

