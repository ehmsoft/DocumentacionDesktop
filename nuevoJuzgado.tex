\section{Nuevo juzgado}
\label{sec:nuevoJuzgado}
Dir\'ijase al men\'u archivo y seleccione \emph{Nuevo}, en este punto se le muestra un listado de los elementos posibles y
all\'i elija la opci\'on \emph{Juzgado}(Fig.\ref{fig:ArchivoNuevoJuzgado}). 
  
\insertImage{Nuevos/Juzgado/MenuArchivo-Nuevo-Juzgado}{Men\'u Archivo>Nuevo>Juzgado}{ArchivoNuevoJuzgado}

Se presentar\'a una pantalla en la que podr\'a introducir toda la informaci\'on
que puede contener un juzgado(Fig.\ref{fig:NuevoJuzgado}). 
  
\insertImage{Nuevos/Juzgado/NuevoJuzgado}{Ventana para crear un juzgado}{NuevoJuzgado}

\subsection{Agregando campos personalizados}
\label{sec:agregarCamposJuzgado}
Recuerde que en cualquier momento puede adicionar un campo personalizado para guardar informaci\'on con un mayor nivel de detalle, este campo se crea presionando el bot\'on Agregar('+') y seleccionando el campo deseado.

\subsection{Modificando campos personalizados}
\label{sec:modificarCamposJuzgado}
Sit\'ue el cursor sobre el campo personalizado que desea modificar y haga click derecho, despu\'es seleccione \emph{Editar},
se presentar\'a una pantalla en la que podr\'a modificar cada caracter\'istica
del campo personalizado. Esta pantalla se profundiza en la secci\'on
\ref{sec:verCampo} (Pag.\pageref{sec:verCampo}).

\subsection{Eliminando campos personalizados}
\label{sec:eliminarCamposJuzgado}
Sit\'ue el cursor sobre el campo personalizado que desea eliminar y haga click derecho, despu\'es seleccione \emph{Eliminar}.
\footnote{El campo Nombre se considera obligatorio y el campo Tel\'efono se
considera importante pero opcional},
cuando termine de ingresarla de click en el bot\'on \emph{Aceptar}.