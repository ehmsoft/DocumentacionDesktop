\section{Nueva plantilla}
\label{sec:nuevaPlantilla}
Dir\'ijase al men\'u archivo y seleccione \emph{Nuevo}, en este punto se muestra un listado de los elementos posibles y
all\'i elija la opci\'on \emph{Plantilla}(Fig.\ref{fig:ArchivoNuevoPlantilla}). 
  
\insertImage{Nuevos/Plantilla/MenuArchivo-Nuevo-Plantilla}{Men\'u Archivo>Nuevo>Plantilla}{ArchivoNuevoPlantilla}

Se presentar\'a una pantalla en la que podr\'a introducir toda la informaci\'on
necesaria para su nueva plantilla, cuando termine de ingresar toda la informaci\'on de click en el bot\'on \emph{Aceptar}. (Fig.\ref{fig:NuevaPlantilla}). 
  
\insertImage{Nuevos/Plantilla/NuevaPlantilla}{Ventana para crear una plantilla}{NuevaPlantilla} 
A continuaci\'on se describen las
caracter\'isticas de cada campo:
\begin{description}
\item[Nombre:]Es un campo de texto usado para diferenciar la plantilla, por lo
tanto se considera obligatorio ingresarlo.
\item[Cliente:]Es una etiqueta no editable que muestra el cliente
seleccionado, para agregarlo solamente haga click sobre este
\footnote{Tambi\'en puede hacerlo con click derecho y seleccionando \emph{cambiar}.}
e inmediatamente se
le mostrar\'a un listado con los clientes existentes del cual podr\'a
seleccionar alguno.
\item[Contraparte:]Es una etiqueta no editable que muestra la contraparte
seleccionada, para cambiarlo solamente haga click sobre este
\footnotemark[\value{footnote}]
e inmediatamente
se le mostrar\'a un listado con las contrapartes existentes del cual podr\'a
seleccionar alguno.
\item[Juzgado:]Es una etiqueta no editable que muestra el juzgado
seleccionado, para cambiarlo solamente haga click sobre este
\footnotemark[\value{footnote}]
e inmediatamente
se le mostrar\'a un listado con los juzgados existentes del cual podr\'a
seleccionar alguno.
\item[Radicado:]Es un campo de texto en el que usted puede ingresar de manera
opcional el radicado que contendr\'a la plantilla.
\item[Radicado \'unico:]Es un campo de texto en el que usted puede ingresar de
manera opcional el radicado \'unico que contendr\'a la plantilla.
\item[Tipo:]Es un campo de texto en el que usted puede ingresar de manera
opcional el tipo de proceso que contendr\'a la plantilla.
\item[Estado:]Es un campo de texto en el que usted puede ingresar de manera
opcional el estado que contendr\'a la plantilla.
\item[Categor\'ia:]Es una etiqueta no editable que muestra la categor\'ia
a la que pertenece el proceso actualmente, para cambiarla solamente haga click
sobre esta
\footnotemark[\value{footnote}]
e inmediatamente
se le mostrar\'a un listado con las categor\'ias existentes del cual podr\'a
seleccionar alguna.
\item[Prioridad:]Es un campo de selecci\'on, con este usted podr\'a asignarle
la prioridad a su nuevo proceso. \'Uselo haciendo ingresando por teclado el valor deseado o dando click en las peque\~nas flechas que tiene el campo para aumentar o disminuir el valor.
\item[Notas:]Es un campo de texto en el que usted puede ingresar de manera
opcional la nota que contendr\'a la plantilla.
\end{description}

\subsection{Agregando campos personalizados}
\label{sec:agregarCamposPlantilla}
Recuerde que en cualquier momento puede adicionar un campo personalizado para guardar informaci\'on con un mayor nivel de detalle, este campo se crea presionando el bot\'on Agregar('+') y seleccionando el campo deseado.

\subsection{Modificando campos personalizados}
\label{sec:modificarCamposPlantilla}
Sit\'ue el cursor sobre el campo personalizado que desea modificar y haga click derecho, despu\'es seleccione \emph{Editar},
se presentar\'a una pantalla en la que podr\'a modificar cada caracter\'istica
del campo personalizado.

\subsection{Eliminando campos personalizados}
\label{sec:eliminarCamposPlantilla}
Sit\'ue el cursor sobre el campo personalizado que desea eliminar y haga click derecho, despu\'es seleccione \emph{Eliminar}.
