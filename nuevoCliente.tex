\section{Nuevo cliente}
\label{sec:nuevoCliente}
Dir\'ijase al men\'u archivo y seleccione \emph{Nuevo}, en este punto se muestra un listado de los elementos posibles y
all\'i elija la opci\'on \emph{Cliente}(Fig.\ref{fig:ArchivoNuevoCliente}). 
  
\insertImage{Nuevos/Cliente/MenuArchivo-Nuevo-Cliente}{Men\'u Archivo>Nuevo>Cliente}{ArchivoNuevoCliente}

Se presentar\'a una pantalla en la que podr\'a introducir toda la informaci\'on
que puede contener un cliente\footnote{El campo Nombre se considera obligatorio y el campo Tel\'efono se
considera importante pero opcional},
cuando termine de ingresarla de click en el bot\'on \emph{Aceptar}.(Fig.\ref{fig:NuevoCliente}). 
  
\insertImage{Nuevos/Cliente/NuevoCliente}{Ventana para crear un Cliente}{NuevoCliente}

\subsection{Agregando campos personalizados}
\label{sec:agregarCamposCliente}
Recuerde que en cualquier momento puede adicionar un campo personalizado para guardar informaci\'on con un mayor nivel de detalle, este campo se crea presionando el bot\'on Agregar('+') y seleccionando el campo deseado.

\subsection{Modificando campos personalizados}
\label{sec:modificarCamposCliente}
Sit\'ue el cursor sobre el campo personalizado que desea modificar y haga click derecho, despu\'es seleccione \emph{Editar},
se presentar\'a una pantalla en la que podr\'a modificar cada caracter\'istica
del campo personalizado.

\subsection{Eliminando campos personalizados}
\label{sec:eliminarCamposCliente}
Sit\'ue el cursor sobre el campo personalizado que desea eliminar y haga click derecho, despu\'es seleccione \emph{Eliminar}.





