\chapter{Consultando la informaci\'on}
\label{sec:consultandoLaInformacion}
La consulta de la informaci\'on es posible realizarla por medio de dos
m\'etodos relacionados:

\begin{description}
  \item[M\'etodo 1:]Por medio de los \emph{listados} en los que se puede ver la
  colecci\'on de elementos del mismo tipo.
  Puede acceder a estas por medio del men\'u \blackberry en
  la pantalla inicial, all\'i se selecciona \textit{Listado} (Fig.
\ref{fig:listadoPantallaInicial}) y en la ventaja emergente elije el tipo de
  elemento que se  desea listar (Fig. \ref{fig:listadoListados})
\insertImage{PantallaInicial/MenuPrincipal-Listado}{Men\'u \emph{Listado} en la
pantalla inicial}{listadoPantallaInicial}
\insertImage{Varios/ListaListado}{Listados de elementos}{listadoListados}
  \item[M\'etodo 2:]Por medio de las \emph{Pantallas de edici\'on} o de
  visualizaci\'on, estas sirven para explorar cada elemento individual.
  Puede acceder a estas estando en un listado haciendo click en cualquier
  elemento  se lanzar\'a la pantalla de \emph{ver (editar)} (Fig.
\ref{fig:verProceso})
\insertImage{Ver/Proceso/Normal}{Pantalla de \emph{Ver}
proceso}{verProceso}

\end{description}

A continuaci\'on se describe el procedimiento para cada elemento usado por la
aplicaci\'on.

\input{listadoActuaciones}
\input{verActuacion}
\input{listadoCategorias}
\input{verCategoria}
\input{listadoCampos}
\input{verCampo}
\input{listadoDemandantes}
\input{verDemandante}
\input{listadoDemandados}
\input{verDemandado}
\input{listadoJuzgados}
\input{verJuzgado}
\input{listadoProcesos}
\input{verProceso}
\input{listadoPlantillas}
\input{verPlantilla}