\chapter{Consultando la informaci\'on}
\label{sec:consultandoLaInformacion}
La consulta de la informaci\'on se realiza por medio de las columnas proporcionadas en la interfaz gr\'afica de la aplicaci\'on, de la siguiente manera

\begin{description}
  \item[M\'etodo 1:]Por medio de los \emph{listados} en los que se puede ver la
  colecci\'on de elementos del mismo tipo.
  
\end{description}

\section{Listados}
La parte de listados es uniforme para casi todos los elementos que contiene la informaci\'on, como se puede observar a simple vista, solamente se necesita dar click en el men\'u principal al elemento del cual desea visualizar la lista, e inmediatamente tendr\'a la lista en la columna de selecci\'on. Con dar click en alg\'un elemento de la columna de selecci\'on se mostrar\'a toda la informaci\'on a la derecha en el \'area de visualizaci\'on.

Nota que en la mayor\'ia de los casos estar\'a disponible una barra de b\'usqueda que puede utilizar para encontrar r\'apidamente alg\'un elemento en especial, lo \'unico que tiene que hacer es empezar a digitar en la barra de b\'usqueda y el programa empezar\'a a encontrar coincidencias. Para volver a listar todos los elementos simplemente borre lo digitado y deje la barra vac\'ia.

Adicionalmente, si se encuentra en la secci\'on de \emph{procesos} o \emph{plantillas}, existir\'a un elemento adicional llamado bot\'on filtrar categor\'ias, que le permitir\'a navegar r\'apidamente entre las categor\'ias que haya creado.