\chapter{Primeros pasos}
\setcounter{page}{4}

\section{Instalaci\'on y configuraci\'on}
Dependiendo del sistema operativo que su ordenador utilice, deber\'a seguir unos pasos para poder instalar y
configurar correctamente su apliaci\'on para un funcionamiento \'optimo. El instalador de la aplicaci\'on lo puede obtener de uno de nuestros asesores comerciales, o lo puede descargar de nuestro sitio web \url{http://www.ehmsoft.com}.

\subsection{\windows}
Una vez obtenido el instalador, debe dar doble click en \'el para que se inicie el asistente de instalaci\'on (Fig.\ref{fig:instaladorWindows}). Lamentamos que el instalador est\'e en idioma ingl\'es pero nuestro software de instalaci\'on viene en este lenguaje. No se preocupe que la aplicaci\'on una vez instalada correr\'a completamente en \emph{castellano}. Como se muestra en la (Fig.\ref{fig:instalador_windows_1}), seleccione la carpeta donde quiere instalar la aplicaci\'on y a continuaci\'on de click en \emph{Next}.
\insertImage{Varios/icono_instalador}{\'Icono del instalador Windows}{instaladorWindows}
\insertImage{Varios/instalador_1_windows}{Pantalla inicial del asistente de instalci\'on}{instalador_windows_1}

En seguida se presentrar\'a una pantalla con una barra de progreso que muestra el avance de la instalaci\'on de la aplicaci\'on (Fig.\ref{fig:instalador_windows_2}). Por favor espere hasta que termine la instalaci\'on.
\insertImage{Varios/instalador_2_windows}{Pantalla de espera del asistente de instalci\'on}{instalador_windows_2}

Al terminar de llenarse completamente la barra de progreso, se mostrar\'a una pantalla de \'exito. A continuaci\'on haga click en el bot\'on \emph{Finish} para concluir la instalaci\'on (Fig.\ref{fig:instalador_windows_3}).
\insertImage{Varios/instalador_3_windows}{Pantalla de \'exito del asistente de instalaci\'on}{instalador_windows_3}
Finalmente en el escritorio de su computador tendr\'a una acceso directo para iniciar el programa \softwareAbogadosDesktop (Fig.\ref{fig:instalador_windows_4}), por favor haga doble click en este \'icono para iniciar el proceso de activaci\'on de la aplicaci\'on.
\insertImage{Varios/instalador_4_windows}{\'Icono en el escritorio del programa}{instalador_windows_4}
\textbf{Nota:} Si tiene alg\'un problema con la instalaci\'on por favor contacte a nuestro equipo de soporte t\'ecnico \mbox{soporte@ehmsoft.com}.

\subsection{\mac}
Una vez obtenido el instalador, debe dar doble click en el archivo \emph{Procesos\_Judiciales\_v1.0.dmg} (Fig.\ref{fig:instaladorMac}). Despu\'es de un momento se habr\'a montado una imagen de disco en su computador la cual est\'a accesible desde el Finder o el escritorio de su \mac, dependiendo de la version de \mac OS instalada. Debe hacer doble click en el \'icono de la imagen de disco que tiene un \'icono de una balanza (Fig.\ref{fig:instalador_mac_1}).
\insertImage{Varios/icono_instalador_mac}{\'Icono del instalador Mac}{instaladorMac}
\insertImage{Varios/icono_instaladorDMG}{\'Icono de la balanza donde debe dar doble click}{instalador_mac_1}
Ahora aparecer\'a una ventana como la de la (Fig.\ref{fig:instalador_mac_2}), lo que debe hacer es coger el \'icono de la balanza que tiene debajo una etiqueta \emph{Procesos Judiciales} y arrastrarlo hasta la carpeta de aplicaciones, siguiendo la direcci\'on indicada por la flecha negra.
\insertImage{Varios/icono_instaladorVentana}{Ventana donde debe arrastrar el \'Icono de aplicaciones}{instalador_mac_2}
La aplicaci\'on estar\'a ahora instalada en su computador. Ahora debe ir a la carpeta \emph{Aplicaciones} y hacer click en ella para ejecutarla y continuar con el proceso de activaci\'on.

\textbf{Nota:} Si tiene alg\'un problema con la instalaci\'on por favor contacte a nuestro equipo de soporte t\'ecnico \mbox{soporte@ehmsoft.com}.
\subsection{Activaci\'on}
Una vez instalada la aplicaci\'on, es necesario activarla para poder trabajar con ella. Esto se hace a trav\'es de un asistente de registro, que lo guiar\'a por los pasos necesarios para activarla. Debe contar con una conexi\'on a internet. Para iniciar la activaci\'on solamente debe iniciar la aplicaci\'on luego de instalarla, e inmediatamente se aparecer\'a el asistente.(Fig.\ref{fig:activacion1}). Lea detenidamente el texto de cada ventana y luego de click en "Siguiente". La segunda ventana le mostrar\'a los t\'erminos y condiciones de uso del programa.(Fig.\ref{fig:activacion2}) Por favor aseg\'urese de leerlos y entenderlos, y \emph{solamente si est\'a de acuerdo} de click en ``Siguiente''. A continuaci\'on se presentar\'a una ventana con un cuadro de texto para digitar el correo que fue registrado al comprar el programa. (Fig.\ref{fig:activacion3}) Si no ha adquirido todav\'ia una licencia, puede hacerlo escribiendo a \mbox{ventas@ehmsoft.com}. Luego de introducirlo, presione ``Finalizar'' y el asistente se conectar\'a a internet para verificar la licencia. Si todo sale bien, ver\'a un cuadro de confirmac\'on. (Fig.\ref{fig:activacion4}) 

\insertImage{Varios/AsistenteRegistro1}{Bienvenida al asistente de registro}{activacion1}
\insertImage{Varios/AsistenteRegistro2}{T\'erminos y condiciones}{activacion2}
\insertImage{Varios/AsistenteRegistro3}{Di\'alogo correo}{activacion3}
\insertImage{Varios/AsistenteRegistro-DialogoActivacionCorrecta}{Confirmaci\'on de activaci\'on correcta}{activacion4}


\section{Conociendo la interfaz de usuario}
La aplicaci\'on utiliza una interfaz sencilla e intuitiva de usar que busca optimizar su flujo de trabajo. Consiste en listas de elementos en forma de columnas que permitir\'an siempre saber d\'onde est\'a ubicado.

La ventana de la aplicaci\'on consta de un men\'u a la izquierda llamado men\'u principal, un elemento central llamado columna de selecci\'on y un elemento en la derecha llamado \'area de visualizaci\'on. (Fig.\ref{fig:general1}). 

\insertImage{Varios/general1Etiquetado}{Ventana de la aplicaci\'on}{general1}

Al seleccionar un elemento en el men\'u principal, se muestra la lista de elementos disponibles en la columna de selecci\'on. La columna de selecci\'on generalmente tiene una barra de b\'usqueda y un bot\'on con el signo '+' llamado bot\'on agregar. Si selecciona Procesos o Plantillas en el men\'u principal adicionalmente tendr\'a visible un bot\'on de filtrar categor\'ias, que sirve mostrar \'unicamente los elementos que pertenecen a una categor\'ia espec\'ifica.
\insertImage{Varios/general2Etiquetado}{Elementos de la columna de selecci\'on}{general2}

Al seleccionar un elemento de la columna de selecc\'on, inmediatamente aparecen en el \'area de visualizaci\'on los detalles del elemento que hemos seleccionado. (Fig.\ref{fig:general3}). 
\insertImage{Varios/general3Etiquetado}{\'Area de visualizaci\'on}{general3}

Lo \'ultimo que debe saber es que en la barra de tareas o notificaciones del sistema operativo aparecer\'a el \'icono de la aplicaci\'on, el cual brinda opciones para mostrar el calendario, ocultar y cerrar la aplicaci\'on
\insertImage{Varios/trayMacEtiquetado}{\'Icono de la barra de tareas en Mac}{trayMac}

\footnote{En caso de problemas con su producto env\'ie un correo
electr\'onico a \mbox{soporte@ehmsoft.com}}

\section{Cosas que debe saber}
La aplicaci\'on cuenta con ciertos elementos que le ayudar\'an a tener una
mejor experiencia con la aplicaci\'on. A continuaci\'on se explicar\'an dichos elementos:

\subsection{Eventos Pr\'oximos}
Es una lista de los eventos(actuaciones) m\'as cercanos a la fecha actual, de los cuales usted debe estar pendiente para evitar perder audiencias y vencimientos, por esta raz\'on se recomienda revisarlos al menos una vez al d\'ia.

\subsection{Calendario}
La aplicaci\'on incluye un calendario y un subistema de alarmas para avisarle y notificarle acerca de las citas que usted programe.

\subsection{Plantillas}
Son elementos \'utiles cuando usted desea crear procesos con
caracter\'isticas similares de forma r\'apida. Una plantilla contiene los mismos
campos de un proceso y en base a esta podr\'a crear procesos con esos campos
pre-ingresados.

\subsection{Campos personalizados}
En la mayor\'ia de las secciones usted podr\'a agregar campos personalizados, estos son
\'utiles cuando usted desea agregar informaci\'on extra a un elemento. Por
ejemplo: Nombre del juez, tel\'efonos adicionales.
