\chapter{Primeros pasos}
\setcounter{page}{4}

\section{Instalaci\'on y configuraci\'on}
Dependiendo del sistema operativo que su ordenador utilice, deber\'a seguir unos pasos para poder instalar y
configurar correctamente su apliaci\'on para un funcionamiento \'optimo. 
\subsection{\windows}
\subsection{\mac}
\subsection{\linux}

\section{Conociendo la interfaz de usuario}
La aplicaci\'on utiliza una interfaz sencilla e intuitiva de usar que busca optimizar su flujo de trabajo. Consiste en listas de elementos en forma de columnas que permitir\'an siempre saber d\'onde est\'a ubicado.

\insertImage{Varios/general1Etiquetado}{Ventana de la aplicaci\'on}{general1}
\insertImage{Varios/general2Etiquetado}{Elementos de la columna de selecci\'on}{general2}
\insertImage{Varios/general3Etiquetado}{\'Area de visualizaci\'on}{general3}
\insertImage{Varios/trayMacEtiquetado}{\'Icono de la barra de tareas en Mac}{trayMac}

\footnote{En caso de problemas con su clave de producto env\'ie un correo
electr\'onico a \mbox{soporte@ehmsoft.com}}


\section{Cosas que debe saber}
La aplicaci\'on cuenta con ciertos elementos que le ayudar\'an a tener una
mejor experiencia con la aplicaci\'on. A continuaci\'on se explicar\'an cuales
son dichos elementos:


\subsection{Plantillas}
Son elementos \'utiles cuando usted desea crear procesos con
caracter\'isticas similares de forma r\'apida. Una plantilla contiene los mismos
campos de un proceso y en base a esta podr\'a crear procesos con esos campos
pre-ingresados.
\subsection{Campos personalizados}
En la mayor\'ia de las secciones usted podr\'a agregar campos personalizados, estos son
\'utiles cuando usted desea agregar informaci\'on extra a un elemento. Por
ejemplo: Nombre del juez, tel\'efonos adicionales.
