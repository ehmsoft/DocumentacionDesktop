\chapter{Primeros pasos}
\setcounter{page}{4}

\section{Instalaci\'on y configuraci\'on}
Dependiendo del sistema operativo que su ordenador utilice, deber\'a seguir unos pasos para poder instalar y
configurar correctamente su apliaci\'on para un funcionamiento \'optimo. 
\subsection{\windows}
\subsection{\mac}
\subsection{\linux}

\section{Conociendo la interfaz de usuario}
La aplicaci\'on utiliza una interfaz sencilla e intuitiva de usar que busca optimizar su flujo de trabajo. Consiste en listas de elementos en forma de columnas que permitir\'an siempre saber d\'onde est\'a ubicado.

La ventana de la aplicaci\'on consta de un men\'u a la izquierda llamado men\'u principal, un elemento central llamado columna de selecci\'on y un elemento en la derecha llamado \'area de visualizaci\'on. (Fig.\ref{fig:general1}). 

\insertImage{Varios/general1Etiquetado}{Ventana de la aplicaci\'on}{general1}

Al seleccionar un elemento en el men\'u principal, se muestra la lista de elementos disponibles en la columna de selecci\'on. La columna de selecci\'on generalmente tiene una barra de b\'usqueda y un bot\'on con el signo '+' llamado bot\'on agregar. Si selecciona Procesos o Plantillas en el men\'u principal adicionalmente tendr\'a visible un bot\'on de filtrar categor\'ias, que sirve mostrar \'unicamente los elementos que pertenecen a una categor\'ia espec\'ifica.
\insertImage{Varios/general2Etiquetado}{Elementos de la columna de selecci\'on}{general2}

Al seleccionar un elemento de la columna de selecc\'on, inmediatamente aparecen en el \'area de visualizaci\'on los detalles del elemento que hemos seleccionado. (Fig.\ref{fig:general3}). 
\insertImage{Varios/general3Etiquetado}{\'Area de visualizaci\'on}{general3}

Lo \'ultimo que debe saber es que en la barra de tareas o notificaciones del sistema operativo aparecer\'a el \'icono de la aplicaci\'on, el cual brinda opciones para mostrar el calendario, ocultar y cerrar la aplicaci\'on
\insertImage{Varios/trayMacEtiquetado}{\'Icono de la barra de tareas en Mac}{trayMac}

\footnote{En caso de problemas con su producto env\'ie un correo
electr\'onico a \mbox{soporte@ehmsoft.com}}


\section{Cosas que debe saber}
La aplicaci\'on cuenta con ciertos elementos que le ayudar\'an a tener una
mejor experiencia con la aplicaci\'on. A continuaci\'on se explicar\'an dichos elementos:

\subsection{Eventos Pr\'oximos}
Es una lista de los eventos(actuaciones) m\'as cercanos a la fecha actual, de los cuales usted debe estar pendiente para evitar perder audiencias y vencimientos, por esta raz\'on se recomienda revisarlos al menos una vez al d\'ia.

\subsection{Calendario}
La aplicaci\'on incluye un calendario y un subistema de alarmas para avisarle y notificarle acerca de las citas que usted programe.

\subsection{Plantillas}
Son elementos \'utiles cuando usted desea crear procesos con
caracter\'isticas similares de forma r\'apida. Una plantilla contiene los mismos
campos de un proceso y en base a esta podr\'a crear procesos con esos campos
pre-ingresados.

\subsection{Campos personalizados}
En la mayor\'ia de las secciones usted podr\'a agregar campos personalizados, estos son
\'utiles cuando usted desea agregar informaci\'on extra a un elemento. Por
ejemplo: Nombre del juez, tel\'efonos adicionales.
