\section{Nueva actuaci\'on}
\label{sec:nuevaActuacion}
Dir\'ijase al men\'u archivo y seleccione \emph{Nuevo}, en este punto se le muestra un listado de los elementos posibles y
all\'i elija la opci\'on \emph{Actuaci\'on}(Fig.\ref{fig:ArchivoNuevoActuacion}). 
  
\insertImage{Nuevos/Actuacion/MenuArchivo-Nuevo-Actuacion}{Men\'u Archivo>Nuevo>Actuaci\'on}{ArchivoNuevoActuacion}

Se presentar\'a una pantalla en la que podr\'a introducir toda la informaci\'on
que puede contener una actuaci\'on (Fig.\ref{fig:NuevaActuacion}). 
  
\insertImage{Nuevos/Actuacion/NuevaActuacion}{Ventana para crear una actuaci\'on}{NuevaActuacion}

 A continuaci\'on se describen las
caracter\'isticas de cada campo:

\begin{description}
\item[Juzgado:]Es una etiqueta no editable que muestra el juzgado
seleccionado, para cambiarlo solamente haga click sobre este
\footnote{Tambi\'en puede hacerlo desplegando el men\'u \blackberry y
seleccionando \emph{cambiar}}
e inmediatamente se le mostrar\'a un listado con los juzgados existentes del
cual podr\'a seleccionar alguno. Este campo se considera obligatorio.
\item[Fecha y fecha pr\'oxima:]Son campos de selecci\'on de fecha, para
modificarlos haga click sobre el elemento que desea editar (la fecha o la hora).
\item[Descripci\'on:]Es un campo de texto en el cual usted describir\'a la
nueva actuaci\'on. Este campo se considera obligatorio.
\end{description}

Cuando termine de ingresar toda la informaci\'on de click en el bot\'on \emph{Guardar}.

\subsection{Creando una cita}
\label{sec:crearCita}
Las actuaciones pueden contener citas, las cuales son agregadas al calendario
de su dispositivo \blackberry, estas tienen la posibilidad de tener una alarma
con la la anticipaci\'on que usted desee.

Despliegue el men\'u \blackberry y seleccione \emph{Agregar cita}, se mostrar\'a una
pantalla en la que podr\'a modificar la descripci\'on o la fecha pr\'oxima,
si selecciona el campo \emph{Alarma}, se activar\'a un campo adicional en el
que podr\'a especificar la anticipaci\'on y en el siguiente campo la duraci\'on
en minutos, horas o d\'ias. Cuando finalice presione \emph{Aceptar}
\footnote{La cita solamente ser\'a agregada o modificada en el calendario cuando
usted guarde la actuaci\'on}.

\subsection{Modificando una cita}
\label{sec:modificarCita}
Si la actuaci\'on ya contiene una cita pero desea modificarla, desplegando el
men\'u \blackberry encontrar\'a la opci\'on \emph{Modificar cita}, se
mostrar\'a una pantalla similar a la usada en la creaci\'on de una cita (Ver.
\ref{sec:crearCita}) en esta podr\'a modificar la informaci\'on de la cita
y seleccionando aceptar confirmar\'a los cambios
\footnotemark[\value{footnote}].

\subsection{Eliminando una cita}
\label{sec:eliminarCita}
Si la actuaci\'on ya contiene una cita pero desea eliminarla, desplegando el
men\'u \blackberry encontrar\'a la opci\'on \emph{Eliminar cita}, se mostrar\'a
un di\'alogo de confirmaci\'on y la cita ser\'a eliminada
\footnotemark[\value{footnote}].