\section{Nueva actuaci\'on}
\label{sec:nuevaActuacion}
Dir\'ijase al men\'u archivo y seleccione \emph{Nuevo}, en este punto se muestra un listado de los elementos posibles y
all\'i elija la opci\'on \emph{Actuaci\'on}(Fig.\ref{fig:ArchivoNuevoActuacion}). 
  
\insertImage{Nuevos/Actuacion/MenuArchivo-Nuevo-Actuacion}{Men\'u Archivo>Nuevo>Actuaci\'on}{ArchivoNuevoActuacion}

Se presentar\'a una pantalla en la que podr\'a introducir toda la informaci\'on
que puede contener una actuaci\'on (Fig.\ref{fig:NuevaActuacion}). 
  
\insertImage{Nuevos/Actuacion/NuevaActuacion}{Ventana para crear una actuaci\'on}{NuevaActuacion}

 A continuaci\'on se describen las
caracter\'isticas de cada campo:

\begin{description}
\item[Juzgado:]Es una etiqueta no editable que muestra el juzgado
seleccionado, para cambiarlo solamente haga click sobre este
\footnotemark[\value{footnote}]
e inmediatamente
se le mostrar\'a un listado con los juzgados existentes del cual podr\'a
seleccionar alguno.
\item[Fecha y fecha pr\'oxima:]Son campos de selecci\'on de fecha, puede introducir la fecha manualmente por teclado o dar click en el peque\~no tri\'angulo del campo para desplegar un calendario donde podr\'a escoger la fecha deseada.

El campo \emph{Fecha} como tal indica el momento donde se gener\'o la actuaci\'on o el estado judicial. El campo \emph{Fecha Pr\'oxima} indica el evento futuro asociado a la actuaci\'on actual, por ejemplo una audiencia programada para dentro de 2 meses. 
\item[Descripci\'on:]Es un campo de texto en el cual usted describir\'a la
nueva actuaci\'on. Este campo se considera obligatorio.
\end{description}

Cuando termine de ingresar toda la informaci\'on de click en el bot\'on \emph{Aceptar}.

\subsection{Agregando campos personalizados}
\label{sec:agregarCamposActuacion}
Recuerde que en cualquier momento puede adicionar un campo personalizado para guardar informaci\'on con un mayor nivel de detalle, este campo se crea presionando el bot\'on Agregar('+') y seleccionando el campo deseado.

\subsection{Modificando campos personalizados}
\label{sec:modificarCamposActuacion}
Sit\'ue el cursor sobre el campo personalizado que desea modificar y haga click derecho, despu\'es seleccione \emph{Editar},
se presentar\'a una pantalla en la que podr\'a modificar cada caracter\'istica
del campo personalizado.

\subsection{Eliminando campos personalizados}
\label{sec:eliminarCamposActuacion}
Sit\'ue el cursor sobre el campo personalizado que desea eliminar y haga click derecho, despu\'es seleccione \emph{Eliminar}.

\subsection{Creando una cita}
\label{sec:crearCita}
Las actuaciones pueden contener citas, las cuales son agregadas al calendario de la aplicaci\'on, estas tienen la posibilidad de tener una alarma
con la la anticipaci\'on que usted desee.

Para agregar una cita de click en la palabra \emph{Cita}, se mostrar\'a una
pantalla en la que podr\'a modificar la descripci\'on o la fecha pr\'oxima,
si selecciona el campo \emph{Alarma}, se activar\'a un campo adicional en el
que podr\'a especificar la anticipaci\'on y en el siguiente campo la duraci\'on
en minutos, horas o d\'ias. Cuando finalice presione \emph{Aceptar}
\footnote{La cita solamente ser\'a agregada o modificada en el calendario cuando
usted guarde la actuaci\'on}.(Fig.\ref{fig:NuevaCitaActuacion}). 
  
\insertImage{Nuevos/Actuacion/NuevaCita}{Ventana para crear una cita}{NuevaCitaActuacion}

\subsection{Modificando una cita}
\label{sec:modificarCita}
Sit\'ue el cursor sobre el campo personalizado que desea modificar y haga click derecho, despu\'es seleccione \emph{Editar}, se
mostrar\'a una pantalla similar a la usada en la creaci\'on de una cita (Ver.
\ref{sec:crearCita}) en esta podr\'a modificar la informaci\'on de la cita
y seleccionando aceptar confirmar\'a los cambios
\footnotemark[\value{footnote}].

\subsection{Eliminando una cita}
\label{sec:eliminarCita}
Si la actuaci\'on ya contiene una cita pero desea eliminarla, simplemente desactive la casilla de verificaci\'on al lado derecho del texto \emph{Cita}, se mostrar\'a
un di\'alogo de confirmaci\'on y la cita ser\'a eliminada
\footnotemark[\value{footnote}].