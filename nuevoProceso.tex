\section{Nuevo Proceso}
\label{sec:nuevoProceso}
Dir\'ijase al men\'u archivo y seleccione \emph{Nuevo}, en este punto se le muestra un listado de los elementos posibles y
all\'i elija la opci\'on \emph{Proceso} (Fig.\ref{fig:ArchivoNuevoProceso}). 
  
\insertImage{Nuevos/Proceso/MenuArchivo-Nuevo-Proceso}{Men\'u Archivo>Nuevo>Proceso}{ArchivoNuevoProceso}

Se presentar\'a una pantalla en la que podr\'a introducir toda la informaci\'on
que puede contener un proceso(Fig.\ref{fig:NuevoProceso}). 
  
\insertImage{Nuevos/Proceso/NuevoProceso}{Ventana para crear un proceso}{NuevoProceso}

 A continuaci\'on se describen las
caracter\'isticas de cada campo:
\begin{description}
\item[Demandante:]Es una etiqueta no editable que muestra el demandante
seleccionado, para agregarlo solamente haga click sobre este
\footnote{Tambi\'en puede hacerlo con click derecho y seleccionando \emph{cambiar}.}
e inmediatamente se
le mostrar\'a un listado con los demandantes existentes del cual podr\'a
seleccionar alguno.
\item[Demandado:]Es una etiqueta no editable que muestra el demandado
seleccionado, para cambiarlo solamente haga click sobre este
\footnotemark[\value{footnote}]
e inmediatamente
se le mostrar\'a un listado con los demandados existentes del cual podr\'a
seleccionar alguno.
\item[Juzgado:]Es una etiqueta no editable que muestra el juzgado
seleccionado, para cambiarlo solamente haga click sobre este
\footnotemark[\value{footnote}]
e inmediatamente
se le mostrar\'a un listado con los juzgados existentes del cual podr\'a
seleccionar alguno.
\item[Radicado:]Es un campo de texto en el que usted puede ingresar de manera
opcional el radicado que contendr\'a el proceso.
\item[Radicado \'unico:]Es un campo de texto en el que usted puede ingresar de
manera opcional el radicado \'unico que contendr\'a el proceso.
\item[Tipo:]Es un campo de texto en el que usted puede ingresar de manera
opcional el tipo que contendr\'a el proceso.
\item[Estado:]Es un campo de texto en el que usted puede ingresar de manera
opcional el estado que contendr\'a el proceso.
\item[Categor\'ia:]Es una etiqueta no editable que muestra la categor\'ia
a la que pertenece el proceso actualmente, para cambiarla solamente haga click
sobre esta
\footnotemark[\value{footnote}]
e inmediatamente
se le mostrar\'a un listado con las categor\'ias existentes del cual podr\'a
seleccionar alguna.
\item[Prioridad:]Es un campo de selecci\'on, con este usted podr\'a asignarle
la prioridad a su nuevo proceso. \'Uselo haciendo ingresando por teclado el valor deseado o dando click en las peque\~nas flechas que tiene el campo para aumentar o disminuir el valor.
\item[Notas:]Es un campo de texto en el que usted puede ingresar de manera
opcional la nota que contendr\'a el proceso.
\end{description}

\subsection{Agregando campos personalizados}
\label{sec:agregarCamposProceso}
Recuerde que en cualquier momento puede adicionar un campo personalizado para guardar informaci\'on con un mayor nivel de detalle, este campo se crea presionando el bot\'on Agregar('+') y seleccionando el campo deseado.

\subsection{Modificando campos personalizados}
\label{sec:modificarCamposProceso}
Sit\'ue el cursor sobre el campo personalizado que desea modificar y haga click derecho, despu\'es seleccione \emph{Editar},
se presentar\'a una pantalla en la que podr\'a modificar cada caracter\'istica
del campo personalizado. Esta pantalla se profundiza en la secci\'on
\ref{sec:verCampo} (Pag.\pageref{sec:verCampo}).

\subsection{Eliminando campos personalizados}
\label{sec:eliminarCamposProceso}
Sit\'ue el cursor sobre el campo personalizado que desea eliminar y haga click derecho, despu\'es seleccione \emph{Eliminar}.

\subsection{Agregando actuaciones}
\label{sec:agregarActuacionesProceso}
De click en la pesta\~na Actuaciones(Fig.\ref{fig:NuevoProceso-Actuacion}), luego presione el bot\'on agregar('+') y se presentar\'a la pantalla para crear una nueva actuaci\'on (Ver.
\ref{sec:nuevaActuacion}).
  
\insertImage{Nuevos/Actuacion/NuevoProceso-ActuacionesOvalo}{Pesta\~na actuaciones}{NuevoProceso-Actuacion}


\subsection{Ver actuaciones}
\label{sec:verActuacionesProceso}
De click en la pesta\~na Actuaciones(Fig.\ref{fig:NuevoProceso-Actuacion}), se le
presentar\'a un listado con las actuaciones pertenecientes a este proceso,
si desea editar o ver m\'as informaci\'on de click derecho en la actuaci\'on y luego seleccione editar. (Ver.
\ref{sec:listadoActuaciones} Pag. \pageref{sec:listadoActuaciones})

\subsection{Eliminando actuaciones}
\label{sec:eliminarActuacionesProceso}
De click en la pesta\~na Actuaciones(Fig.\ref{fig:NuevoProceso-Actuacion}), se le
presentar\'a un listado con las actuaciones pertenecientes a este proceso,
de click derecho en la actuaci\'on y luego seleccione eliminar.

Cuando termine de ingresar toda la informaci\'on de click en el bot\'on \emph{Aceptar}.

