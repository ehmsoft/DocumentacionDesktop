\section{Nuevo Proceso}
\label{sec:nuevoProceso}
Dir\'ijase al men\'u archivo y seleccione \emph{Nuevo}, en este punto se le muestra un listado de los elementos posibles y
all\'i elija la opci\'on \emph{Proceso} (Fig.\ref{fig:ArchivoNuevoProceso}). 
  
\insertImage{Nuevos/Proceso/MenuArchivo-Nuevo-Proceso}{Men\'u Archivo>Nuevo>Proceso}{ArchivoNuevoProceso}

Se presentar\'a una pantalla en la que podr\'a introducir toda la informaci\'on
que puede contener un proceso(Fig.\ref{fig:NuevoProceso}). 
  
\insertImage{Nuevos/Proceso/NuevoProceso}{Ventana para crear un proceso}{NuevoProceso}

 A continuaci\'on se describen las
caracter\'isticas de cada campo:
\begin{description}
\item[Demandante:]Es una etiqueta no editable que muestra el demandante
seleccionado, para cambiarlo solamente haga click sobre este
\footnote{Tambi\'en puede hacerlo desplegando el men\'u \blackberry y
seleccionando \emph{cambiar}}
e inmediatamente se
le mostrar\'a un listado con los demandantes existentes del cual podr\'a
seleccionar alguno.
\item[Demandado:]Es una etiqueta no editable que muestra el demandado
seleccionado, para cambiarlo solamente haga click sobre este
\footnotemark[\value{footnote}]
e inmediatamente
se le mostrar\'a un listado con los demandados existentes del cual podr\'a
seleccionar alguno.
\item[Juzgado:]Es una etiqueta no editable que muestra el juzgado
seleccionado, para cambiarlo solamente haga click sobre este
\footnotemark[\value{footnote}]
e inmediatamente
se le mostrar\'a un listado con los juzgados existentes del cual podr\'a
seleccionar alguno.
\item[Radicado:]Es un campo de texto en el que usted puede ingresar de manera
opcional el radicado que contendr\'a el proceso.
\item[Radicado \'unico:]Es un campo de texto en el que usted puede ingresar de
manera opcional el radicado \'unico que contendr\'a el proceso.
\item[Tipo:]Es un campo de texto en el que usted puede ingresar de manera
opcional el tipo que contendr\'a el proceso.
\item[Estado:]Es un campo de texto en el que usted puede ingresar de manera
opcional el estado que contendr\'a el proceso.
\item[Categor\'ia:]Es una etiqueta no editable que muestra la categor\'ia
a la que pertenece el proceso actualmente, para cambiarla solamente haga click
sobre esta
\footnotemark[\value{footnote}]
e inmediatamente
se le mostrar\'a un listado con las categor\'ias existentes del cual podr\'a
seleccionar alguna.
\item[Prioridad:]Es un campo de selecci\'on, con este usted podr\'a asignarle
la prioridad a su nuevo proceso. \'Uselo haciendo click sobre este y
despu\'es seleccionando el valor deseado.
\item[Notas:]Es un campo de texto en el que usted puede ingresar de manera
opcional la nota que contendr\'a el proceso.
\end{description}

\subsection{Agregando campos personalizados}
\label{sec:agregarCamposProceso}
Despliegue el men\'u \blackberry y seleccione \emph{Agregar campo personalizado}, se
presentar\'a un listado con los campos existentes, all\'i elija el campo
personalizado que desea a\~nadir al proceso.

\subsection{Modificando campos personalizados}
\label{sec:modificarCamposProceso}
Sit\'ue el cursor sobre el campo personalizado que desea modificar, y despliegue el
men\'u \blackberry, despu\'es seleccione \emph{Modificar},
se presentar\'a una pantalla en la que podr\'a modificar cada caracter\'istica
del campo personalizado. Esta pantalla se profundiza en la secci\'on
\ref{sec:verCampo} (Pag.\pageref{sec:verCampo}).

\subsection{Eliminando campos personalizado}
\label{sec:eliminarCamposProceso}
Sit\'ue el cursor sobre el campo personalizado que desea eliminar, y despliegue el
men\'u \blackberry, despu\'es seleccione \emph{Eliminar}.

\subsection{Agregando actuaciones}
\label{sec:agregarActuacionesProceso}
Despliegue el men\'u \blackberry y seleccione \emph{Nueva actuaci\'on} y se
presentar\'a la pantalla para crear una nueva actuaci\'on (Ver.
\ref{sec:nuevaActuacion}).

\subsection{Ver actuaciones}
\label{sec:verActuacionesProceso}
Despliegue el men\'u \blackberry y seleccione \emph{Ver actuaciones}, se le
presentar\'a un listado con las actuaciones pertenecientes a este proceso y
haciendo click en cada una podr\'a ver la informaci\'on que contienen. (Ver.
\ref{sec:listadoActuaciones} Pag. \pageref{sec:listadoActuaciones})

\subsection{Eliminando actuaciones}
\label{sec:eliminarActuacionesProceso}
Despliegue el men\'u \blackberry y seleccione \emph{Ver actuaciones}, se le
presentar\'a un listado con las actuaciones pertenecientes a este proceso y
desplegando el men\'u \blackberry nuevamente selecciona la opci\'on
\emph{Eliminar}.

Cuando termine de ingresar toda la informaci\'on de click en el bot\'on \emph{Guardar}.

