\section{Nuevo campo personalizado}
\label{sec:nuevoCampo}
Dir\'ijase al men\'u archivo y seleccione \emph{Nuevo}, all\'i elija la opci\'on \emph{Campo personalizado}, en este punto se le muestra un listado de los elementos, de click en el elemento que desee crear (Fig.\ref{fig:ArchivoNuevoCampo}). 
  
\insertImage{Nuevos/Campo/MenuArchivo-Nuevo-Campo-CampoProceso}{Men\'u Archivo>Nuevo>Campo Personalizado}{ArchivoNuevoCampo}

Se presentar\'a una pantalla en la que podr\'a introducir el nombre del nuevo
campo personalizado (Fig.\ref{fig:NuevoCampo}). 
  
\insertImage{Nuevos/Campo/NuevoCampoPersonalizado}{Ventana para crear un campo personalizado}{NuevoCampo}

\footnote{El campo Nombre se considera obligatorio},
as\'i como tres
alternativas:
\begin{description}
\item[Obligatorio:]Es un campo de selecci\'on, que le permite determinar si el
campo que est\'a siendo creado, cuando se agregue a un elemento siempre
debe contener informaci\'on.
\item[Longitud m\'axima:]Es un campo para insertar n\'umeros, este tiene la
funci\'on de indicar que el campo que est\'a siendo creado, cuando se agregue a un elemento siempre debe tener informaci\'on de una longitud
m\'axima del valor introducido.
\footnote{Al ingresar 0 o dejarlo vac\'io indicar\'a que este par\'ametro
ser\'a ignorado}
\item[Longitud m\'inima:]Es un campo para insertar n\'umeros, este tiene la
funci\'on de indicar que el campo que est\'a siendo creado, cuando se agregue a un elemento siempre debe tener informaci\'on de una longitud
m\'inima del valor introducido.
\footnotemark[\value{footnote}]
\end{description}

Cuando termine de ingresar toda la informaci\'on de click en el bot\'on \emph{Aceptar}.